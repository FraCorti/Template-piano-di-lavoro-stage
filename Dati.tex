%----------------------------------------------------------------------------------------
%   USEFUL COMMANDS
%----------------------------------------------------------------------------------------

\newcommand{\dipartimento}{Dipartimento di Matematica ``Tullio Levi-Civita''}

%----------------------------------------------------------------------------------------
% 	USER DATA
%----------------------------------------------------------------------------------------

% Data di approvazione del piano da parte del tutor interno; nel formato GG Mese AAAA
% compilare inserendo al posto di GG 2 cifre per il giorno, e al posto di 
% AAAA 4 cifre per l'anno
\newcommand{\dataApprovazione}{26/06/2019}

% Dati dello Studente
\newcommand{\nomeStudente}{Francesco}
\newcommand{\cognomeStudente}{ Corti}
\newcommand{\matricolaStudente}{1142525}
\newcommand{\emailStudente}{francesco.corti@studenti.unipd.it}
\newcommand{\telStudente}{+ 39 348 80 48 083}

% Dati del Tutor Aziendale
\newcommand{\nomeTutorAziendale}{Davide}
\newcommand{\cognomeTutorAziendale}{Baldo}
\newcommand{\emailTutorAziendale}{davide.baldo@zextras.com}
\newcommand{\telTutorAziendale}{}
\newcommand{\ruoloTutorAziendale}{}

% Dati dell'Azienda sede legale
\newcommand{\ragioneSocAzienda}{Zextras srl}
\newcommand{\indirizzoAzienda}{Sede Legale: Piazza San Nazaro in Brolo 15, 20122 Milano}
\newcommand{\sitoAzienda}{https://www.zextras.com/}
\newcommand{\emailAzienda}{info@pec.zextras.com}
\newcommand{\partitaIVAAzienda}{P.IVA: IT03695830244}
\newcommand{\sedeOperativa}{Sede operativa:
via dell'industria 8, 36040 Torri di Quartesolo (VI)}

% Dati del Tutor Interno (Docente)
\newcommand{\titoloTutorInterno}{Prof.}
\newcommand{\nomeTutorInterno}{Davide}
\newcommand{\cognomeTutorInterno}{Bresolin}

\newcommand{\prospettoSettimanale}{
     % Personalizzare indicando in lista, i vari task settimana per settimana
     % sostituire a XX il totale ore della settimana
    \begin{itemize}
        \item \textbf{Prima Settimana (40 ore)}
        \begin{itemize}
        	\item Cos'è Zimbra, cos'è Zextras; 
        	\item Come sono organizzati i moduli del prodotto e come comunicano tra di loro.
        \end{itemize}
        \item \textbf{Seconda Settimana (40 ore)} 
        \begin{itemize}
            \item Studio e benchmarking delle soluzioni sviluppate dai competitor;
            \item Definire la lista delle funzionalità che si
andranno ad implementare e la relativa UX.
        \end{itemize}
        \item \textbf{Terza Settimana (40 ore)} 
        \begin{itemize}
            \item Definire il set di API necessario per lo sviluppo dell'applicativo e verificarne la presenza;
            \item Procedere all'implementazione, se necessario, delle API mancanti.
        \end{itemize}
        \item \textbf{Quarta Settimana (40 ore)} 
        \begin{itemize}
            \item Realizzazione di alcuni proof of concept con diverse tecnologie;
            \item Decisione del framework da adottare per lo sviluppo dell'applicativo.
        \end{itemize}
        \item \textbf{Quinta Settimana (40 ore)} 
        \begin{itemize}
        \item Inizio sviluppo dell'applicativo;
            \item Preparazione ambiente test.
        \end{itemize}
        \item \textbf{Sesta Settimana (40 ore)} 
        \begin{itemize} 
        	\item Sviluppo applicativo;       	
            \item Scrittura test.
        \end{itemize}
        \item \textbf{Settima Settimana (40 ore)} 
        \begin{itemize}
        	\item Terminazione sviluppo applicativo;
        	\item Inizio collaudo applicativo.            
        \end{itemize}
        \item \textbf{Ottava Settimana (40 ore)} 
        \begin{itemize}
        	\item Collaudo applicativo;
            \item Preparazione di un pacchetto di installazione;
            \item Test manuali end2end in ambiente
controllato;
			\item Presentazione della soluzione al management aziendale.
        \end{itemize}
    \end{itemize}
}

% Indicare il totale complessivo (deve essere compreso tra le 300 e le 320 ore)
\newcommand{\totaleOre}{}

\newcommand{\obiettiviObbligatori}{
	 \item \underline{\textit{O01}}: prototipo dell'applicativo funzionante in ambiente Linux (Ubuntu);
	 \item \underline{\textit{O02}}: documentazione chiara ed appropriata mirata all'utilizzo e al mantenimento del prodotto;	
	  \item \underline{\textit{O03}}: realizzazione di test unitari per tutte le funzionalità sviluppate;
	  \item \underline{\textit{O04}}: realizzazione di una presentazione che esponga i punti salienti di quanto realizzato.
}

\newcommand{\obiettiviDesiderabili}{
	 \item \underline{\textit{D01}}: integrazione completa con il file manager del Sistema Operativo;
	 \item \underline{\textit{D02}}: gestione delle revisioni e del concurrent locking dei documenti.
}

\newcommand{\obiettiviFacoltativi}{
	 \item \underline{\textit{F01}}: realizzazione di una versione in Markdown della documentazione, da inserire nella repository del progetto.
}